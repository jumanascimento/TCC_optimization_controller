\section*{\fontsize{14}{16}\selectfont
RESUMO
}


% Texto do resumo totalmente JUSTIFICADO
\justifying
Este trabalho aborda o desenvolvimento e validação de um algoritmo de otimização para tolerância à falha em veículos subaquáticos remotamente operados (ROVs), especificamente no BlueROV2 Standard. O objetivo principal é manter a navegabilidade do veículo em casos de falha total de um ou mais \textit{thrusters}, através da redistribuição otimizada das forças entre os propulsores remanescentes. A metodologia envolveu simulações computacionais utilizando o Gazebo Ignition e ROS 2 Control, testando diferentes cenários de falhas em trajetórias de movimento (\textit{surge}, \textit{sway}, \textit{heave} e \textit{yaw}). Os resultados demonstram que o algoritmo é eficaz na manutenção da navegabilidade do ROV, especialmente quando as combinações mínimas de \textit{thrusters} são respeitadas, melhorando significativamente a velocidade média e distância percorrida mesmo em condições de falha. O trabalho contribui para o avanço da segurança operacional de ROVs em aplicações industriais submarinas.

\vspace{0.5cm}

\noindent\textbf{PALAVRAS-CHAVE:} ROV; otimização; tolerância a falhas; \textit{thrusters}; veículos subaquáticos
