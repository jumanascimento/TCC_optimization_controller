\documentclass[../main.tex]{subfiles}
\graphicspath{{\subfix{../images/}}}
\begin{document}

Nesta seção, são apresentados os resultados obtidos a partir dos experimentos conduzidos para avaliar o desempenho do algoritmo de otimização proposto na metodologia. Os resultados são organizados em tabelas que ilustram o impacto do algoritmo na navegabilidade do ROV em diferentes cenários de falhas.

\subsection{Combinações Mínimas de Thrusters}

Na Tabela~\ref{tab:comb_thrusters}, são apresentadas as combinações mínimas de thrusters necessários para manter a navegabilidade do ROV em cada direção de movimento (\textit{surge, sway, heave, yaw}), considerando os diferentes casos de falhas simuladas.

\begin{table}[htbp]
\centering
\small
\sffamily
\caption{Combinações mínimas de thrusters por direção (\textit{surge, sway, heave, yaw})}
\vspace{0.3em}
\begin{tabular}{ll}
\toprule
\textbf{Direção} & \textbf{Combinação Mínima de Thrusters} \\
\midrule
\textbf{Surge} & Dois \textit{thrusters} paralelos ou diagonais \\
\textbf{Sway} & Dois \textit{thrusters} laterais ou diagonais \\
\textbf{Heave} & Dois \textit{thrusters} verticais \\
\textbf{Yaw} & Dois \textit{thrusters} diagonais \\
\bottomrule
\end{tabular}
\label{tab:comb_thrusters}
\end{table}
\begin{center}
{\footnotesize Fonte: Autores.}
\end{center}

A partir da Tabela~\ref{tab:comb_thrusters}, é possível observar que alguns casos de falhas caem fora das combinações mínimas necessárias para manter a navegabilidade do ROV em determinadas direções. Isso indica que, em tais situações, o algoritmo de otimização não conseguiu redistribuir as forças de maneira eficaz para compensar a ausência dos thrusters falhos, resultando na incapacidade do ROV de se mover conforme desejado.

\subsection{Resultados da Direção Surge}

A Tabela~\ref{tab:erro_direcao_surge} apresenta os erros médios obtidos após as repetições dos testes para a direção \textit{Surge}. Os erros foram obtidos comparando a trajetória ideal sem o algoritmo de otimização com as trajetórias realizadas em cada caso de falha, tanto com o algoritmo de otimização (COM) quanto sem o algoritmo (SEM).

% Tabela formatada conforme normas SENAI CIMATEC/ABNT
\begin{table}[H]
\centering
\small % Fonte menor conforme orientações
\sffamily % Fonte sans-serif (Arial/Helvetica)

% Título acima da tabela conforme orientações
\caption{Erro medio por Direção (\textit{Surge})}

\vspace{0.5em}

% Tabela sem linhas verticais, usando booktabs
\begin{tabular}{p{0.3cm}lcccccccc}
\toprule
\textbf{Caso} & &\textbf{Vel (m/s)} & \textbf{Distância (m)} & \textbf{X (m)} & \textbf{Y (m)} & \textbf{Z (m)} & \textbf{yaw (°)} \\
\midrule
\multirow{2}{*}{\textbf{1}} & COM & 0,005$\pm$0,031 & 0,448$\pm$0,101 &-7,198$\pm$0,105 & 1,040$\pm$0 & 0,013$\pm$0,014 & -360$\pm$0 \\
 & SEM & 0,096$\pm$0,005 & 3,475$\pm$0,117 & 7,105$\pm$0,108 & -1,116$\pm$0 & 0,014$\pm$0,008 & -360$\pm$0 \\
\addlinespace[0.2em]
\multirow{2}{*}{\textbf{2}} & COM & -0,025$\pm$0,004	&-0,024$\pm$0,115 & -0,068$\pm$0,120	&-0,01$\pm$0,001	&0,041$\pm$0,003	&-0,1$\pm$0\\
 & SEM & 0,091$\pm$0,008	&3,630$\pm$0,247	&3,592$\pm$0,247	&0,00$\pm$0	&0,038$\pm$0,004	&0$\pm$0\\
\addlinespace[0.2em]
\multirow{2}{*}{\textbf{3}} & COM & 0,037$\pm$0,010 &1,716$\pm$0,122 &5,237$\pm$1,159 &-3,770$\pm$0,027 &-0,088$\pm$0,180 &-11,4$\pm$0,089 \\
 & SEM &0,033$\pm$0,013 &1,684$\pm$0,328 &3,003$\pm$0,282 &-3,820$\pm$0,152 &0,006$\pm$0,023 &-11,6$\pm$0,515 \\
\addlinespace[0.2em]
\multirow{2}{*}{\textbf{4}} & COM &-0,006$\pm$0,013 &-0,171$\pm$0,132 &-0,209$\pm$0,116 &-0,49$\pm$0,088 &-0,002$\pm$0,028 &-6,9$\pm$0,612 \\
 & SEM & 0,027$\pm$0,004 & 0,939$\pm$0,252 & 5,823$\pm$0,089 & -3,39$\pm$0 & 0,012$\pm$0,005 & -360$\pm$0 \\
\addlinespace[0.2em]
\multirow{2}{*}{\textbf{5}} & COM & -0,007$\pm$0,010 & -0,008$\pm$0,181 & -0,008$\pm$0,177 & -0,38$\pm$0,012 & 0,012$\pm$0,022 & -6,18$\pm$0,098 \\
 & SEM & 0,025$\pm$0,007 & 1,129$\pm$0,013 & 5,862$\pm$0,017 & -3,39$\pm$0 & 0,016$\pm$0,013 & -360$\pm$0 \\
\bottomrule
\end{tabular}
\label{tab:erro_direcao_surge}
\end{table}

\vspace{0.3em}
\begin{center}
{\footnotesize Fonte: Autores.}
\end{center}

Além disso, é possível observar que nos casos onde a combinação mínima de thrusters não é atendida, o algoritmo de otimização não consegue compensar adequadamente a falta dos propulsores, resultando em erros mais elevados. Isso evidencia a importância de uma configuração adequada dos thrusters para garantir a eficácia do algoritmo de otimização na manutenção da navegabilidade do ROV. Mesmo com o impacto na trajetória, o algoritmo de otimização ainda apresenta melhoras em relação a velocidade media do ROV e a distancia percorrida, quando comparado ao caso sem otimização. É possivel visualizar o impacto disposição insatisfatoria de \textit{thrusters} no caso~1, onde apesar de apresentar um melhora na velocidade media e na distancia percorrida no caso com otimização em relação ao caso sem otimização, ainda apresenta um erro de~360° na orientação \textit{yaw}, indicando a impossibilidade do ROV manter a trajetoria desejada em~\textit{Surge}.

A figura \ref{surge_trajectories} ilustra as trajetórias do ROV para a direção \textit{Surge} nos casos 0 (ideal), 1, 2 e 5, com o algoritmo de otimização em ação. Observa-se que no caso 1, onde a combinação mínima de thrusters não é atendida, o ROV apresenta uma trajetória significativamente desviada da ideal, evidenciando a incapacidade do algoritmo de otimização em compensar a falha dos thrusters, devido a limitação de \textit{hardware} do ROV. Em contraste, no caso 2, onde a combinação mínima é satisfeita, o ROV consegue seguir uma trajetória muito mais próxima da ideal, demonstrando a eficácia do algoritmo quando as condições mínimas são atendidas. O caso 5 também mostra uma melhora considerável em relação ao caso 1, embora ainda apresente desvios notáveis devido à falha do thruster.

\begin{figure}[H]
  \centering
  \caption{Trajetórias do ROV em \textit{Surge} para os casos 0, 1, 2 e 5.}

  % 4 subfiguras em uma linha com tamanho reduzido
  \begin{subfigure}{0.45\textwidth}
    \centering
    \includegraphics[width=\textwidth]{images/x_ideal.png}
    \caption{\textit{Surge} Trajetória Ideal}
    \label{fig:rov_front}
  \end{subfigure}\hspace{0.2em}%
  \begin{subfigure}{0.45\textwidth}
    \centering
    \includegraphics[width=\textwidth]{images/x_caso_1.png}
    \caption{\textit{Surge} Caso 1}
    \label{fig:rov_side}
  \end{subfigure}\hspace{0.2em}%
  \begin{subfigure}{0.45\textwidth}
    \centering
    \includegraphics[width=\textwidth]{images/x_caso_2.png}
    \caption{\textit{Surge} Caso 2}
    \label{fig:rov_iso}
  \end{subfigure}\hspace{0.2em}%
  \begin{subfigure}{0.45\textwidth}
    \centering
    \includegraphics[width=\textwidth]{images/x_caso_5.png}
    \caption{\textit{Surge} Caso 5}
    \label{fig:rov_bottom}
  \end{subfigure}
  
  \par\smallskip
  \small Fonte: Autores.
  \label{surge_trajectories}
\end{figure}

A figura~\ref{surge_4_comparation} exemplifica a melhora realizada pela otimização quando a configuração minima de \textit{thrusters} é respeitada, sendo possível garantir que o ROV siga a trajetória desejada com desvios mínimos. Relacionando a trajetória apresentada na figura~\ref{surge_4_comparation} com os dados do caso 4 na tabela \ref{tab:erro_direcao_surge}, é possível observar que o erro medio em \textit{X} é de apenas -0,209 m quando o algoritmo de otimização está ativo, enquanto sem o algoritmo, o erro médio aumenta significativamente para 5,823 m, que juntamente com o erro em \textit{yaw} de apenas 6,9° com otimização e de 360° sem a otimização, indicam uma incapacidade do ROV de manter a trajetória desejada sem a otimização. Isso demonstra a eficácia do algoritmo de otimização em melhorar a navegabilidade do ROV em situações de falhas nos thrusters, desde que as combinações mínimas sejam atendidas. 


\begin{figure}[H]
  \centering
  \caption{Comparação da trajetória do ROV em \textit{Surge} para o caso 4 com e sem otimização.}

  % 4 subfiguras em uma linha com tamanho reduzido
  % \begin{subfigure}{0.45\textwidth}
  %   \centering
  %   \includegraphics[width=\textwidth]{images/y_caso_ideal.png}
  %   \caption{\textit{Sway} Trajetória Ideal}
  %   \label{fig:rov_front}
  % \end{subfigure}\hspace{0.2em}%
  \begin{subfigure}{0.45\textwidth}
    \centering
    \includegraphics[width=\textwidth]{images/x_caso_4_com_qp.png}
    \caption{\textit{Surge} Caso 4 com otimização}
    \label{fig:rov_side}
  \end{subfigure}\hspace{0.2em}%
  \begin{subfigure}{0.45\textwidth}
    \centering
    \includegraphics[width=\textwidth]{images/x_caso_4_sem_qp.png}
    \caption{\textit{Surge} Caso 4 sem otimização}
    \label{fig:rov_iso}
  \end{subfigure}\hspace{0.2em}%
  
  
  \par\smallskip
  \small Fonte: Autores.
  \label{surge_4_comparation}
\end{figure}


A tabela~\ref{tab:erro_direcao_sway} apresenta os resultados dos erros medios medidos na direção \textit{sway}. Na trajetória \textit{sway}, o caso 1 atende às combinações minimas de \textit{thrusters}, resultando em erros significativamente menores em comparação com os outros casos, em relação a trajetoria, mas ainda é possível observar a ação da otimização na velocidade e na distancia percorrida.
O caso 2 não atende às combinações minimas de \textit{thrusters} para a movimentação em \textit{sway}, resultando em erros maiores na trajetória, mesmo com o algoritmo de otimização ativo. No entanto, o algoritimo ainda melhora a distancia percorrida e a velocidade media, ou seja mesmo com a falha, o ROV  mantem a capacidade de se movimentar com uma velocidade proxima a ideal, percorrendo distancias parecidas, apresentando um erro medio de apenas -0,006 m/s em velocidade e 0,248 m em distancia percorrida, quando com o otimizador ativado, em contraste com erros medios de -0,052 m/s e 1,897 m respectivamente, quando o otimizador está desativado.


\begin{table}[H]
\centering
\small % Fonte menor conforme orientações
\sffamily % Fonte sans-serif (Arial/Helvetica)

% Título acima da tabela conforme orientações
\caption{Erro medio por Direção (\textit{Sway})}

\vspace{0.5em}

% Tabela sem linhas verticais, usando booktabs
\begin{tabular}{lcccccccc}
\toprule
\textbf{Caso} & & \textbf{Vel (m/s)} & \textbf{Distância (m)} & \textbf{X (m)} & \textbf{Y (m)} & \textbf{Z (m)} & \textbf{yaw (°)} \\
\midrule
\multirow{2}{*}{\textbf{1}} & COM & 0,006 & -0,100 &-0,051 & -0,100 & 0,006 & -0,44 \\
 & SEM & 0,055 & 1,870 & 0,844 & 1,694 & 0,001 & 10,72 \\
\addlinespace[0.2em]
\multirow{2}{*}{\textbf{2}} & COM & -0,006	&0,248 & 0,714	&5,500	&0,026	&-333,8\\
 & SEM & 0,052	&1,897	&0,470	&5,450	&0,016	&-333,8\\
\addlinespace[0.2em]
\multirow{2}{*}{\textbf{3}} & COM & 0,003 &-0,156 &-3,885 &2,40 &-0,015 &11,2 \\
 & SEM &0,004 &0,123 &-3,368&	2,46	&-0,001	&10,92 \\
\addlinespace[0.2em]
\multirow{2}{*}{\textbf{4}} & COM &-0,005 &-0,007 &-0,313	&0,10	&0,011	&-6,26 \\
 & SEM & 0,009 & 0,334 & -1,013	&4,28	&0,003	&-333,8 \\
\addlinespace[0.2em]
\multirow{2}{*}{\textbf{5}} & COM & -0,008& -0,206& -0,767	&-0,07	&0,004	&-7,46 \\
 & SEM & 0,010 & 0,484 & -1,011 & 4,28 & 0,012 & -333,8 \\
\bottomrule
\end{tabular}
\label{tab:erro_direcao_sway}
\end{table}

\vspace{0.3em}
\begin{center}
{\footnotesize Fonte: Autores.}
\end{center}

O otimizador demonstra sua eficiencia principalmente no ambito da velocidade media e distancia percorrida, mesmo em casos onde a combinação minima de \textit{thrusters} não é atendida. Como o \textit{BlueRov2 Standerd}, possui uma configuração de 6 \textit{thrusters}, o otimizador encontra limitações na redistribuição das forças quando múltiplos \textit{thrusters} falham, o que  causa uma dificuldade em manter a trajétoria ideal, sendo possível visualizar erros significativos em \textit{Yaw} nos casos que não atendem as combinações minimas para aquela direção.
Nos casos que atendem as combinações minimas, o otimizador é capaz de diminuir o error medio em \textit{yaw}, que é causado pelas falhas dos \textit{thrusters}, mantendo assim a capacidade do ROV de seguir a trajetoria desejada.

O caso 3 se destaca por apresentar erros muito proximos entre os cenários com e sem otimização em qualquer trajetoria, indicando que a falha diagonal representada pelo caso três tem a capacidade de se autocompensar devido ao posicionamento dos \textit{thrusters} no ROV, o que minimiza o impacto da falha na capacidade de movimentação do veículo. Ainda que o impacto neste caso seja menos é possível obter melhorias significativas na velocidade media e distancia percorrida com o otimizador ativado. O maior fator de melhoria observado para o algoritimo de otimização é onservado tanto na distancia percorrida quanto na velocidade media, mantendo a capacidade do ROV de manter a velocidade o mais proxima do ideal possível e preservando também a capacidade de percorrer distancias similares ao caso ideal, mesmo com a falha de \textit{thrusters}.
Em relação a manutenção da trajetória ideal, o otimizador enfrenta dificuldades devido as limitações de \textit{hardware} do ROV, essas limitações são refletidas nos casos que não respeitam as combinações minimas de \textit{thrusters}, e na trajetoria \textit{heave}. A combinação minima exigida para que haja o funcioamento ideal da trajetória \textit{heave} são de 2 \textit{thrusters} como indicado na tabela \ref{tab:comb_thrusters}, mas o ROV utilizado nos testes apresenta somente dois \textit{thrusters} verticais em sua estrutura, impossíbilitando que seja realizada a otimização no caso de falha de um dos \textit{thrusters} verticais, o que impossibilita a movimentação ideal na direção \textit{heave}. Em decorrencia dessa limitação física do ROV, os resultados obtidos para a trajetória \textit{heave} não são apresentados nesta seção.

Na figura~\ref{sway_trajectories} são ilustradas as trajetórias do ROV para a direção \textit{Sway} nos casos 0 (ideal), 1, 2 e 5, com o algoritmo de otimização em ação. Assim como para a trajetória \textit{Surge}, observa-se que no caso em que a falha não obedece a combinação mínima de thrusters (caso 2), o ROV apresenta uma trajetória significativamente desviada da ideal, evidenciando a incapacidade do algoritmo de otimização em compensar a falha dos thrusters devido a limitação de \textit{hardware} do ROV. Em contraste, no caso 1, onde a combinação mínima é satisfeita, o ROV consegue seguir uma trajetória muito mais próxima da ideal, demonstrando a eficácia do algoritmo quando as condições mínimas são atendidas. O caso 5 também mostra uma melhora considerável em relação ao caso 2, embora ainda apresente desvios notáveis devido à falha do thruster.


\begin{figure}[H]
  \centering
  \caption{Trajetórias do ROV em \textit{Sway} para os casos 0, 1, 2 e 5.}

  % 4 subfiguras em uma linha com tamanho reduzido
  \begin{subfigure}{0.45\textwidth}
    \centering
    \includegraphics[width=\textwidth]{images/y_caso_ideal.png}
    \caption{\textit{Sway} Trajetória Ideal}
    \label{fig:rov_front}
  \end{subfigure}\hspace{0.2em}%
  \begin{subfigure}{0.45\textwidth}
    \centering
    \includegraphics[width=\textwidth]{images/y_caso_1.png}
    \caption{\textit{Sway} Caso 1}
    \label{fig:rov_side}
  \end{subfigure}\hspace{0.2em}%
  \begin{subfigure}{0.45\textwidth}
    \centering
    \includegraphics[width=\textwidth]{images/y_caso_2.png}
    \caption{\textit{Sway} Caso 2}
    \label{fig:rov_iso}
  \end{subfigure}\hspace{0.2em}%
  \begin{subfigure}{0.45\textwidth}
    \centering
    \includegraphics[width=\textwidth]{images/y_caso_5.png}
    \caption{\textit{Sway} Caso 5}
    \label{fig:rov_bottom}
  \end{subfigure}
  
  \par\smallskip
  \small Fonte: Autores.
  \label{sway_trajectories}
\end{figure}

Na figura \ref{sway_4_comparation} é possível observar o efeito do algoritmo de otimização na trajetória do ROV para o caso 4, sendo possível visualizar a melhora significativa na trajetória do ROV quando o algoritmo de otimização está presente quando comparamos a trajetória feita com a trajetória ideal apresentada na figura \ref{sway_trajectories}, apresentando somente um pequeno desvio em relação a trajetória ideal, enquanto sem o algoritmo o ROV não seria capaz de manter a trajetória desejada, apresentando um desvio considerável em relação a trajetória ideal.

\begin{figure}[H]
  \centering
  \caption{Comparação da trajetória do ROV em \textit{Sway} para o caso 4 com e sem otimização.}

  % 4 subfiguras em uma linha com tamanho reduzido
  % \begin{subfigure}{0.45\textwidth}
  %   \centering
  %   \includegraphics[width=\textwidth]{images/y_caso_ideal.png}
  %   \caption{\textit{Sway} Trajetória Ideal}
  %   \label{fig:rov_front}
  % \end{subfigure}\hspace{0.2em}%
  \begin{subfigure}{0.45\textwidth}
    \centering
    \includegraphics[width=\textwidth]{images/y_caso_4_com_qp.png}
    \caption{\textit{Sway} Caso 4 com otimização}
    \label{fig:rov_side}
  \end{subfigure}\hspace{0.2em}%
  \begin{subfigure}{0.45\textwidth}
    \centering
    \includegraphics[width=\textwidth]{images/y_caso_4_sem_qp.png}
    \caption{\textit{Sway} Caso 4 sem otimização}
    \label{fig:rov_iso}
  \end{subfigure}\hspace{0.2em}%
  
  
  \par\smallskip
  \small Fonte: Autores.
  \label{sway_4_comparation}
\end{figure}


Na tabela~\ref{tab:erro_direcao_yaw} são apresentados os resultados dos erros medios obtidos na direção \textit{yaw}. Considerando a combinação ideal da trajetória \textit{yaw} como sendo a combinação diagonal de \textit{thrusters}, os resultados reforçam que a combinação do caso 3 obteve os melhores resultados nos testes. Contudo, o impacto do algoritmo de otimizaçao na trajetória \textit{yaw} é menos pronunciado em comparação com as outras direções, devido as caracteristicas da movimentação rotacional do ROV juntamente com as limitações fisicas do veiculo em situação de falhas nos \textit{thrusters}. Como metrica foi observada a capacidade do ROV completar o numero de voltas mais proximo do ideal, a capacidade de se mantar na orientação correta sem se desviar nas outras direções e a velocidade media de rotação. Assim nesse contexto foi observado que os casos que apresentavam somente um \textit{thruster} em falha (casos 4 e 5) apresentaram resultados mais satisfatórios, pois além de atenderam a combinação minima de \textit{thrusters} para a movimentação em \textit{yaw}, apresentam um \textit{thruster} a mais em funcionamento, o que possibilita uma melhor redistruibuição das forças, nesses casos apesar de apresentarem resultados melhores que os demais casos (1 e 2) a presença da otimização não foi capaz de melhorar os resultados quando comparados ao cenário sem otimização, indicando que a limitação física do ROV em termos de \textit{thrusters} disponíveis impacta significativamente a eficácia do algoritmo de otimização na manutenção da navegabilidade em situações de falhas.


% Tabela formatada conforme normas SENAI CIMATEC/ABNT
\begin{table}[H]
\centering
\small % Fonte menor conforme orientações
\sffamily % Fonte sans-serif (Arial/Helvetica)

% Título acima da tabela conforme orientações
\caption{Erro medio por Direção (\textit{Yaw})}

\vspace{0.5em}

% Tabela sem linhas verticais, usando booktabs
\begin{tabular}{lcccccccc}
\toprule
\textbf{Caso} & & \textbf{Vel (°/s)} & \textbf{N° de Voltas} & \textbf{X (m)} & \textbf{Y (m)} & \textbf{Z (m)} & \textbf{yaw (°)} \\
\midrule
\multirow{2}{*}{\textbf{1}} & COM & 92,360 & 8,020 &-1,026 & -1,101 & 0,006 & 2897,22 \\
 & SEM & 48,020 & 4,180 & -0,910 & -186,728 & 0,002 & 1502,48 \\
\addlinespace[0.2em]
\multirow{2}{*}{\textbf{2}} & COM & 99,78 & 9,020 &	-1,440&	-1,31	&0,008&	3138,84\\
 & SEM & 49,120&	4,3	&-0,672	&-0,67	&0,001	&1550,44\\
\addlinespace[0.2em]
\multirow{2}{*}{\textbf{3}} & COM & 44,880&	3,900	&-0,001&	0,00&	0,008	&1399,74 \\
 & SEM &49,300&	4,34	&-0,001	&0,00	&0,012	&1556,72 \\
\addlinespace[0.2em]
\multirow{2}{*}{\textbf{4}} & COM &41,060&	2,720	&-0,069&	-0,06	&-0,001	&1235,02 \\
 & SEM & 25,060&	2,26	&-0,376	&-0,38	&0,007	&806,1 \\
\addlinespace[0.2em]
\multirow{2}{*}{\textbf{5}} & COM & 41,360&	3,520&	-0,066&	-0,07	&0,002	&1265,32 \\
 & SEM & 25,580&	2,14	&-0,375&	-0,38&	-0,001&	775,62 \\
\bottomrule
\end{tabular}
\label{tab:erro_direcao_yaw}
\end{table}

\vspace{0.3em}
\begin{center}
{\footnotesize Fonte: Autores.}
\end{center}

A figura~\ref{yaw_trajectories} ilustra as trajetórias do ROV para a direção \textit{Yaw} nos casos 0 (ideal), 1, 3 e 5, com o algoritmo de otimização em ação. Observa-se que, assim como mostrado na tabela~\ref{tab:erro_direcao_yaw}, o otimizador não apresenta uma melhora sifnificativa na trajétoria do ROV em \textit{Yaw}, sendo incapaz de manter a trajétoria no caso 1, e apesar de conseguir completar a trajetória nos casos 3 e 5, ainda apresenta desvios consideráveis em relação à trajetória ideal, não sendo capaz de completar o mesmo numero de voltas, o que pode ser visto na densidade das linhas nas figuras. Relacionando ainda com a tabela~\ref{tab:erro_direcao_yaw}, o algoritimo não foi capaz de melhorar a valocidade media de rotação mesmo nos casos que atendem as combinações minimas de \textit{thrusters} para a movimentação em \textit{Yaw} (casos 3 e 5), sendo capaz de garantir que o rov complete a trajetorias, mas não no mesmo periodo de temo e com uma velocidade cignificativamente reduzida. 



\begin{figure}[H]
  \centering
  \caption{Trajetórias do ROV em \textit{Yaw} para os casos 0, 1, 3 e 5.}
  \begin{subfigure}{0.45\textwidth}
    \centering
    \includegraphics[width=\textwidth]{images/yaw_caso_0.png}
    \caption{\textit{Yaw} Trajetória Ideal}
    \label{fig:rov_front}
  \end{subfigure}\hspace{0.2em}%
  \begin{subfigure}{0.45\textwidth}
    \centering
    \includegraphics[width=\textwidth]{images/yaw_caso_1.png}
    \caption{\textit{Yaw} Caso 1}
    \label{fig:rov_side}
  \end{subfigure}\hspace{0.2em}%
  \begin{subfigure}{0.45\textwidth}
    \centering
    \includegraphics[width=\textwidth]{images/yaw_caso_3.png}
    \caption{\textit{Yaw} Caso 3}
    \label{fig:rov_iso}
  \end{subfigure}\hspace{0.2em}%
  \begin{subfigure}{0.45\textwidth}
    \centering
    \includegraphics[width=\textwidth]{images/yaw_caso_5.png}
    \caption{\textit{Yaw} Caso 5}
    \label{fig:rov_bottom}
  \end{subfigure}
  
  \par\smallskip
  \small Fonte: Autores.
  \label{yaw_trajectories}
\end{figure}

\end{document}

